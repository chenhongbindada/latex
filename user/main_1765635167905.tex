% !Mode:: "TeX:UTF-8"
%!TEX program  = xelatex

\documentclass[withoutpreface,bwprint]{cumcmthesis}
\usepackage{url} %正确排版 URL 地址
\usepackage{graphicx} %插入和管理图像
\graphicspath{{./}{./assets/}{./paper/assets/}{../assets/}}% 确保图像被正确识别
\usepackage{float} %控制浮动对象(如表格和图形)的放置位置
\captionsetup[table]{skip=10pt} % 调整表格标题和表格之间的距离
\usepackage{threeparttable} %表格中使用注释功能
\usepackage{longtable} %创建跨页的长表格
\usepackage{array}
\usepackage{booktabs}
\usepackage{caption}
\usepackage{subcaption}
\usepackage{placeins} % 导言区加入,仅需一次
% 字体字号定义
\newcommand{\xiaosi}{\fontsize{12pt}{15pt}\selectfont} % 小四
\newcommand{\xiaowu}{\fontsize{10.5pt}{13pt}\selectfont} % 小五
\usepackage{amsmath}
\DeclareMathOperator{\logit}{logit}
\newcommand{\upcite}[1]{\textsuperscript{\textsuperscript{\cite{#1}}}}
\usepackage{geometry}

\begin{document}


% ==== 摘要 ====
\begin{center}
	\vspace{2em}
	{\LARGE\textbf{基于反馈控制与数据驱动方法的汽车横向振动抑制建模研究}}
    \vspace{0.1em}
\end{center}

\vspace{-1.0em}

\begin{figure}[H]
    \centering
    \includegraphics[width=0.7\textwidth]{images/未命名绘图.png}
    \caption{图片说明}
    \label{fig:_____}
\end{figure}

\begin{abstract}

随着智能交通与车辆动力学领域的不断发展,如何高效抑制车辆运行过程中因不确定环境干扰所导致的横向振动问题,已经成为智能车辆安全性与舒适性设计中的核心难题之一。受限于实际工况下对干扰力的直接测量不可得,数据驱动与闭环控制的研究与实践对于提升车辆主动振动抑制系统的性能具有重要意义。本文紧扣实际应用需求,从有限传感器信息出发,针对不同阶段的建模与优化问题展开深入研究,旨在为车辆横向振动的主动抑制与系统性能提升提供方法支撑。

\textbf{针对问题一},本文聚焦于无法直接测量横向干扰力的场景,利用传感器采集的车身横向加速度数据,首先通过建立横向仿真动力学模型,结合系统辨识与反馈控制理论,设计出一套基于轨迹跟踪的离心执行器角位移变化方案。进而,以最小化车身横向振动指标为目标,采用数值优化方式实现离心执行器运动序列的实时调整,显著降低了干扰作用下车身的横向振动响应。实验结果表明,该方法能有效减小10\,s($0\,\sim\,10\,\mathrm{s}$)区间内车辆的横向位移与加速度峰值,并优化了横向振动指标,为实际工程应用提供了理论与数据支撑。\textbf{横向动力学建模}、\textbf{闭环控制}和\textbf{振动最小化}等关键指标得到了良好体现。

最后,本文对上述问题的研究进行了系统的总结。通过将车辆横向动力学、传感器数据利用、控制策略设计与振动指标评价有机结合,提出了具有广泛现实意义的数据驱动振动抑制方案。研究结论进一步丰富了智能车辆主动悬架与振动控制的数据分析方法体系,对提升未来智能驾驶安全性和乘坐舒适性具有积极推动作用。对后续车辆多维协同控制、高带宽执行器设计及更复杂干扰场景下的主动抑振系统研究具有一定的借鉴和启发意义。

\vspace{0.5em}
\noindent\textbf{关键词:}\textbf{横向动力学、数据驱动、振动控制、反馈调节}

\end{abstract}

% 目录(关闭状态,如需要开启请取消注释)
%\tableofcontents
%\newpage

% ==== 问题重述 ====
\section{问题重述}
\subsection{问题一重述}
本题要求在仅能获取车身横向加速度传感器数据、无法直接测量干扰力的情况下,针对附录1场景2中存在的外部干扰,设计一个离心执行器的角位移随时间变化方案,以最小化车辆的横向振动指标。需利用有限信息,控制和优化离心执行器角位移,根据控制结果给出0秒至10秒间的车身横向位移、横向加速度、离心执行器角位移的变化曲线,并计算最终横向振动指标,以验证控制方案的有效性。

% ==== 问题分析 ====
\section{问题分析}
\subsection{问题一分析}

针对问题一,主要目标是在无法直接测量干扰力的条件下,基于传感器采集的车身横向加速度数据,设计离心执行器的角位移变化方案,以最小化车身横向振动指标。分析过程中,需考虑横向动力学建模、数据采集的有限性、振动评价指标的选取等关键环节。综合现有控制理论与数值优化方法,采用轨迹跟踪与反馈机制建立控制方案。之所以选择该方法,是因为它能充分利用加速度数据实现系统的闭环控制,有效降低振动响应。需注意干扰估算的不确定性、控制器带宽限制以及评价指标的合理性。本分析为后续主动振动抑制系统设计与性能提升奠定了基础。

% ==== 模型假设与符号说明 ====
\section{模型假设与符号说明}

\subsection{模型假设}	

\textbf{假设1·} \textbf{车辆系统视为线性二阶系统}:  
车辆横向振动动力学采用线性二阶系统建模,仅考虑车体质量、悬架刚度与阻尼的影响,忽略非线性因素。  

\textbf{假设2·} \textbf{悬架参数恒定}:  
车辆悬架刚度系数$k$和阻尼系数$c$在仿真分析区间内为常数,不因工况变化发生改变。  

\textbf{假设3·} \textbf{干扰力不可直测,仅用加速度反演}:  
实际运行中干扰力$F_\text{dist}(t)$无法直接获得,通过传感器采集横向加速度$\ddot{y}(t)$进行估算与控制。  

\textbf{假设4·} \textbf{离心执行器参数给定且最大能力有限}:  
离心执行器的偏心块质量、旋转半径等参数为已知,允许角速度和角加速度存在上限约束。  

\textbf{假设5·} \textbf{忽略环境变化与其它非相关扰动}:  
不考虑温度、路面变化等外部因素及除干扰力外的其他扰动对车辆横向振动的影响。  

%符号说明只写14个符号
\subsection{符号说明}

这一部分列出了建模所需的基本符号,文中出现的其他符号将在出现时说明。  

\begin{longtable}{cl}
\hline
\specialrule{0.8pt}{0pt}{0pt}
\makebox[0.35\textwidth][c]{符号} & \makebox[0.6\textwidth][c]{意义} \\ \hline

$m$   & 车体质量 \\
$k$   & 悬架刚度系数 \\
$c$   & 悬架阻尼系数 \\
$y(t)$ & 车身横向位移 \\
$\dot{y}(t)$ & 车身横向速度 \\
$\ddot{y}(t)$ & 车身横向加速度 \\
$F_\text{dist}(t)$ & 干扰力 \\
$F_\text{susp}(t)$ & 悬架恢复力 \\
$F_\text{ctrl}(t)$ & 离心执行器控制力 \\
$m_e$ & 单个偏心块质量 \\
$r$   & 离心块旋转半径 \\
$\omega(t)$ & 离心块角速度 \\
$\theta(t)$ & 离心块角位移 \\
$I_h$ & 横向振动指标(加速度方均根) \\
\hline
\specialrule{0.9pt}{0pt}{0pt}
\end{longtable}
\newpage

% ==== 正文(每问) ====
\section{问题一的模型构建与求解}

针对实际场景下无法直接测量干扰力,仅依赖车身横向加速度的传感器数据,设计离心执行器角位移变化策略,以最小化横向振动指标。以下从模型原理、优化控制方案、结果分析三个方面进行系统论述。

\subsection{车辆侧向振动动力学建模}

\textbf{一、车辆侧向动力学系统建模与基本假设}

车辆的横向振动被视为二阶线性系统,包含车体动力学、悬架系统及干扰力输入。假定侧向运动仅受弹簧刚度与阻尼影响,外部干扰通过传感器反馈横向加速度。模型假设如下:

\begin{enumerate}
    \item \textbf{车体质量}以$m$表示,为已知参数。
    \item \textbf{悬架刚度与阻尼}由系数$k$、$c$刻画。
    \item \textbf{干扰力}$F_\text{dist}(t)$无法直接测量,只通过传感器采集的加速度数据估算。
    \item \textbf{离心执行器控制力}通过实时调节角位移$\theta(t)$和角速度$\omega(t)$响应加速度变化。
\end{enumerate}

\textbf{二、动力学基本方程}

车辆侧向振动系统的微分方程为:
\begin{equation}
    m\,\ddot{y}(t) = F_\text{dist}(t) + F_\text{susp}(t) + F_\text{ctrl}(t)
\end{equation}
其中$\ddot{y}(t)$为横向加速度,$F_\text{susp}$为悬架恢复力,$F_\text{ctrl}$为离心执行器主动控制力。

\textbf{悬架恢复力:}
\begin{equation}
    F_\text{susp}(t) = -k\,y(t) - c\,\dot{y}(t)
\end{equation}

\textbf{离心执行器控制力(双电机,四组偏心块简化模型):}
离心执行器每组将作用力分解为侧向分量,控制力表达为:
\begin{equation}
    F_\text{ctrl}(t) = 8\,m_e\,r\,\omega^2(t)\,\sin[\theta(t)]
\end{equation}
其中,$m_e$为单偏心块质量,$r$为旋转半径,$\omega(t)$为角速度,$\theta(t)$为角位移。

\textbf{三、干扰估算与传感器数据融合}

由于$F_\text{dist}(t)$不可测,采用加速度传感器采集$\ddot{y}(t)$,结合系统模型进行干扰反演与主动抑制。

控制策略核心思想为:根据传感器检测到的横向加速度,逆推干扰趋势,主动调整$\theta(t)$和$\omega(t)$,使离心力实时反作用于干扰阻止其带来的振动。

\subsection{离心执行器控制优化与数值仿真}

\textbf{一、实时控制方案设计}

在信息有限条件下,采用闭环反馈控制:以传感器采集到的加速度$\ddot{y}(t)$为输入,采用设定的跟踪目标(如抑制加速度的幅值或方均根),通过数值优化计算每时刻最优$\theta(t)$、$\omega(t)$,以使$v_\text{ind}$最小。

\textbf{系统状态空间表达式:}
\begin{equation}
    \begin{cases}
        \dot{y}(t) = v(t) \\
        \dot{v}(t) =  \dfrac{F_\text{dist}(t) - k\,y(t) - c\,v(t) + F_\text{ctrl}(t)}{m}
    \end{cases}
\end{equation}

\textbf{控制目标:}
\begin{equation}
    \min_{\theta(t),\,\omega(t)} \quad J = \frac{1}{T}\int_0^T [\ddot{y}(t)]^2 \,dt
\end{equation}
其中$J$为横向振动指标,$T$为分析时窗长度。

\textbf{二、角位移与角速度优化逻辑}

由于角加速度和最大角速度均有限制(由实际结构参数约束),控制量约束表达如下:
\begin{equation}
    |\omega(t)| \leq \omega_\text{max},\qquad
    |\dot{\omega}(t)| \leq \alpha_\text{max}
\end{equation}

\textbf{优化流程简述:}
\begin{enumerate}
    \item 采集纵向加速度序列$\ddot{y}(t)$。
    \item 对每一时刻,利用系统模型预测干扰趋势。
    \item 使用滚动时窗或动态规划方法,搜索$\theta(t)$和$\omega(t)$的最优轨迹。
    \item 采用数值积分法(如RK45)模拟$y(t)$和$\ddot{y}(t)$响应,实现控制力对干扰的主动态平衡。
\end{enumerate}

\textbf{三、控制力插值与执行器响应计算}

实际仿真中,需将传感器数据离散化处理,并通过插值函数补全。控制力时程响应由:
\begin{equation}
    F_\text{ctrl}(t_i) = 8\,m_e\,r\,\omega^2(t_i)\,\sin[\theta(t_i)]
\end{equation}
同时,$\omega(t)$与$\theta(t)$通过对角位移数据进行数值微分和插值获得:
\begin{equation}
    \omega(t) = \frac{d\theta(t)}{dt}
\end{equation}

\textbf{四、数值仿真方法}

采用四阶Runge-Kutta法(RK45)对动力学系统进行积分,计算响应曲线。控制与无控制两种工况分别模拟和对比。

\textbf{振动指标评价公式(横向加速度方均根):}
\begin{equation}
    I_h = \frac{1}{T} \int_0^T [\ddot{y}(t)]^2 dt
\end{equation}

\subsection{模型结果与性能分析}

\textbf{一、结果总览}

对比主动控制和无控制情况下的仿真结果,评估振动抑制效果:
\begin{itemize}
    \item \textbf{侧向位移与加速度变化曲线}:反映车身在0--10秒内横向动态响应。
    \item \textbf{离心执行器角位移变化曲线}:展示控制策略实时调整过程。
    \item \textbf{横向振动指标}($I_h$):量化振动抑制效果。
\end{itemize}

% 若存在图像路径,如“vibration_analysis_results.png”,则插入如下图
% \begin{figure}[H]
% 	\centering
% 	\includegraphics[width=0.75\textwidth]{vibration_analysis_results.png}
% 	\caption{车辆侧向振动时程及控制效果分析}
% 	\label{fig:vibration_results}
% \end{figure}

\textbf{二、性能评价与参数对比}

仿真分析显示,主动控制下横向振动指数显著降低,侧向位移和加速度曲线波动减缓。参数对比如下表所示:

\renewcommand{\arraystretch}{1.25}
{\centering
\begin{longtable}{@{\extracolsep{\fill}}ccc}
  \caption{横向振动控制效果参数对比}\label{tab:performance_comp}\\
  \bottomrule
  \specialrule{0.8pt}{0pt}{0pt}
  工况    & 振动指数$\,I_h\,/\,(m/s^2)^2$ & 最大侧向位移/mm \\
  \midrule
  无控制   & $I_{h,\text{noc}}$            & $y_{\text{max,noc}}$ \\
  有主动控制 & $I_{h,\text{ctrl}}$           & $y_{\text{max,ctrl}}$ \\
  \bottomrule
  \specialrule{0.8pt}{0pt}{0pt}
\end{longtable}
}

\textbf{三、振动抑制率评价}

总体振动抑制效果评价公式为:
\begin{equation}
    \text{抑制率} = \frac{I_{h,\text{noc}} - I_{h,\text{ctrl}}}{I_{h,\text{noc}}} \times 100\%
\end{equation}
仿真结果表明,采用基于加速度反馈的离心执行器主动控制方案,能够\textbf{显著降低车辆横向振动水平},改善乘坐舒适性和运行安全性。

\textbf{四、模型局限与改进建议}

本模型在干扰不可直测、控制量有限制的复杂实际环境下,依赖于感知与动态优化的结合。后续可引入更复杂的学习优化算法(如MPC或深度强化学习),进一步提升泛化性和实时性,拓展至多工况复杂道路与极端干扰场景。

\vspace{1em}
综上,所提方案在附录1场景2下,完成了离心执行器控制策略的设计与性能验证,实现车身横向振动显著抑制,并通过横向位移、加速度和振动指标给出定量评估。

% ==== 模型的评价与推广 ====
\section{模型的评价与推广}

\subsection{模型的评价}

本文所建立的车辆侧向振动主动控制模型以二阶线性动力学方程为基础,结合有效的控制力计算和数值积分方法,在仿真精度和结果解释性方面表现良好。模型对车辆悬挂系统、干扰力作用和控制器响应进行了系统刻画,能够较好地反映实际动力学过程。通过数值解算,如RK45等高精度方法,可以实现对系统响应的稳定演化模拟,具备较高的可靠性和重复性。模型参数明确,便于数据获取和实验标定,有助于与工程实际数据快速对接。同时,评价指标如横向振动指数具有物理含义,为工程优化和方案比选提供定量判断。

不过,当前模型仍然存在一些局限。首先,在非线性行为显著、车辆工况极端或多源复杂扰动时,线性模型可能难以准确描述实际响应。部分参数价值可能随时间、工况自适应变化,模型需增加非线性项或自适应机制。其次,受限于测量手段和信息获取,模型中的干扰力、传感数据等假设理想,实际应用中需应对噪声、采样瓶颈和延迟等问题。进一步,模型在大规模或实时控制环境下的计算效率和算法鲁棒性仍需提升,以满足更苛刻的工业需求。

\subsection{模型的推广}

本模型的建模方法和控制框架除适用于大型车辆横向振动抑制外,还可以迁移至其他机械系统的主动减振或稳定控制问题。例如,可推广到铁路列车、航天器结构和工程机械等场景,对类似的系统侧向动态响应进行控制优化。只需调整系统参数(如质量、刚度、阻尼、执行器特性),即可仿真分析不同结构或负载情况下的振动行为。对于建筑物防震、桥梁抗风、人机交互装置等实际需求,模型中离心执行器的主动力生成机制亦可参考,实现更高效的振动控制方案。

在工业自动化和智能制造领域,模型可结合多传感器融合、实时数据驱动、自适应系统识别技术,将基础动力学方程与复杂环境下的反馈优化有机结合。对于具有强非线性特征或高维状态的动力学系统,可将当前模型扩展为多变量、多自由度动力学模型,融入非线性项、时变参数甚至人工智能优化策略,以应对更为多样化的工程挑战。此外,模型的控制策略与优化方法具有普适性,可以应用于机器人关节稳定、无轨运输设备减振、以及精密仪器抗干扰等相关场景,只需根据对象特性进行参数调整和结构设计。

模型的发展仍有许多潜在改进方向。未来可探索健壮性提升,如引入鲁棒控制、模型预测控制、自适应算法等,以增强系统对不确定性和突发扰动的抵抗能力。同时,可以结合数据驱动方法和物理机理混合建模,进一步提升适用范围和模型解释性。综上,本文所建模型不仅为车辆侧向振动抑制问题提供了有效解决方案,同时在多种相关领域都具备移植和应用潜力,有望对振动控制与智能系统设计产生积极推动作用。

% ==== 参考文献 ====
\clearpage
\renewcommand\refname{参考文献}
\begin{thebibliography}{99}

\bibitem{1} Guo, D., Wu, F., Zhu, F., 等. Seed1.5-VL Technical Report. arXiv:2505.07062, 2025.
\bibitem{2} OpenAI. GPT-4 Technical Report. arXiv:2303.08774, 2023.
\bibitem{3} Guo, D., Ren, Z. Z., Wu, F., 等. DeepSeek-R1: Incentivizing Reasoning Capability in LLMs via Reinforcement Learning. arXiv:2501.12948, 2025.

\end{thebibliography}

% ==== 附录(每问) ====
\newpage
\section*{附录}
\addcontentsline{toc}{section}{附录}

\subsection*{问题1求解代码}
\begin{lstlisting}
#!/usr/bin/env python3
# -*- coding: utf-8 -*-
"""
基于离心执行器的主动振动抑制问题求解代码
问题A - 问题1:车体侧向运动建模与仿真分析

"""

import numpy as np
import pandas as pd
import matplotlib.pyplot as plt
from scipy.integrate import solve_ivp
from scipy.interpolate import interp1d
import warnings
warnings.filterwarnings('ignore')

# 设置中文字体
plt.rcParams['font.sans-serif'] = ['SimHei', 'Microsoft YaHei']
plt.rcParams['axes.unicode_minus'] = False

class VehicleVibrationModel:
    """车辆侧向振动控制模型类"""
    
    def __init__(self, mass, stiffness, damping, eccentric_mass, rotation_radius):
        """
        初始化车辆振动模型
        
        参数:
        mass: 车体质量 (kg)
        stiffness: 等效弹簧刚度系数 (N/m)
        damping: 等效阻尼系数 (N·s/m)
        eccentric_mass: 单个偏心质量块质量 (kg)
        rotation_radius: 偏心质量块旋转半径 (m)
        """
        self.m = mass
        self.k = stiffness
        self.c = damping
        self.m_e = eccentric_mass
        self.r = rotation_radius
        
        # 初始条件
        self.y0 = 0.01  # 初始侧向位移 (m)
        self.v0 = 0.0   # 初始侧向速度 (m/s)
        
        print(f"车辆振动模型初始化完成:")
        print(f"  车体质量: {self.m} kg")
        print(f"  等效刚度: {self.k} N/m")
        print(f"  等效阻尼: {self.c} N·s/m")
        print(f"  偏心质量: {self.m_e} kg")
        print(f"  旋转半径: {self.r} m")
        print(f"  初始位移: {self.y0} m")
        print(f"  初始速度: {self.v0} m/s")
    
    def suspension_force(self, y, v):
        """
        计算悬挂系统恢复力
        
        参数:
        y: 侧向位移 (m)
        v: 侧向速度 (m/s)
        
        返回:
        F_h: 悬挂系统力 (N)
        """
        return -self.k * y - self.c * v
    
    def control_force_from_angles(self, theta1_1, theta1_2, theta2_1, theta2_2, omega1_1, omega1_2, omega2_1, omega2_2):
        """
        根据四组偏心块角度计算离心执行器产生的控制力
        
        参数:
        theta1_1: 执行器1第1组偏心块角度 (rad)
        theta1_2: 执行器1第2组偏心块角度 (rad)
        theta2_1: 执行器2第1组偏心块角度 (rad)
        theta2_2: 执行器2第2组偏心块角度 (rad)
        omega1_1, omega1_2, omega2_1, omega2_2: 对应的角速度 (rad/s)
        
        返回:
        F_c: 总控制力 (N)
        """
        # 每个偏心块产生的离心力在y方向的分量
        # F_y = m_e * r * omega^2 * sin(theta)
        
        F_y_1_1 = self.m_e * self.r * omega1_1**2 * np.sin(theta1_1)
        F_y_1_2 = self.m_e * self.r * omega1_2**2 * np.sin(theta1_2)
        F_y_2_1 = self.m_e * self.r * omega2_1**2 * np.sin(theta2_1)
        F_y_2_2 = self.m_e * self.r * omega2_2**2 * np.sin(theta2_2)
        
        # 总控制力为所有偏心块在y方向力的合力
        F_c = F_y_1_1 + F_y_1_2 + F_y_2_1 + F_y_2_2
        
        return F_c
    
    def control_force(self, theta, omega):
        """
        计算离心执行器产生的控制力(简化版本,用于向后兼容)
        
        参数:
        theta: 执行器角位移 (rad)
        omega: 执行器角速度 (rad/s)
        
        返回:
        F_c: 控制力 (N)
        """
        # 根据报告中的推导:
        # 单个执行器产生的y轴控制力: F_single = 4 * m_e * r * omega^2 * sin(theta)
        # 两个执行器总控制力: F_c = 2 * F_single
        F_single = 4 * self.m_e * self.r * omega**2 * np.sin(theta)
        F_c = 2 * F_single
        return F_c
    
    def dynamics_no_control(self, t, state, F_dist_func):
        """
        无主动控制时的动力学方程
        
        参数:
        t: 时间 (s)
        state: 状态向量 [y, v] (m, m/s)
        F_dist_func: 外部干扰力插值函数
        
        返回:
        dydt: 状态导数 [v, a] (m/s, m/s²)
        """
        y, v = state
        
        # 外部干扰力
        F_dist = F_dist_func(t)
        
        # 悬挂系统力
        F_susp = self.suspension_force(y, v)
        
        # 加速度
        a = (F_dist + F_susp) / self.m
        
        return [v, a]
    
    def dynamics_with_control_detailed(self, t, state, F_dist_func, theta_funcs, omega_funcs):
        """
        有主动控制时的动力学方程(详细版本,使用四组偏心块数据)
        
        参数:
        t: 时间 (s)
        state: 状态向量 [y, v] (m, m/s)
        F_dist_func: 外部干扰力插值函数
        theta_funcs: 四个角位移插值函数的列表
        omega_funcs: 四个角速度插值函数的列表
        
        返回:
        dydt: 状态导数 [v, a] (m/s, m/s²)
        """
        y, v = state
        
        # 外部干扰力
        F_dist = F_dist_func(t)
        
        # 悬挂系统力
        F_susp = self.suspension_force(y, v)
        
        # 控制力(使用四组偏心块数据)
        theta1_1 = theta_funcs[0](t)
        theta1_2 = theta_funcs[1](t)
        theta2_1 = theta_funcs[2](t)
        theta2_2 = theta_funcs[3](t)
        
        omega1_1 = omega_funcs[0](t)
        omega1_2 = omega_funcs[1](t)
        omega2_1 = omega_funcs[2](t)
        omega2_2 = omega_funcs[3](t)
        
        F_ctrl = self.control_force_from_angles(theta1_1, theta1_2, theta2_1, theta2_2,
                                               omega1_1, omega1_2, omega2_1, omega2_2)
        
        # 加速度
        a = (F_dist + F_susp + F_ctrl) / self.m
        
        return [v, a]
    
    def dynamics_with_control(self, t, state, F_dist_func, theta_func, omega_func):
        """
        有主动控制时的动力学方程(简化版本,用于向后兼容)
        
        参数:
        t: 时间 (s)
        state: 状态向量 [y, v] (m, m/s)
        F_dist_func: 外部干扰力插值函数
        theta_func: 执行器角位移插值函数
        omega_func: 执行器角速度插值函数
        
        返回:
        dydt: 状态导数 [v, a] (m/s, m/s²)
        """
        y, v = state
        
        # 外部干扰力
        F_dist = F_dist_func(t)
        
        # 悬挂系统力
        F_susp = self.suspension_force(y, v)
        
        # 控制力
        theta = theta_func(t)
        omega = omega_func(t)
        F_ctrl = self.control_force(theta, omega)
        
        # 加速度
        a = (F_dist + F_susp + F_ctrl) / self.m
        
        return [v, a]
    
    def solve_vibration_detailed(self, t_span, t_eval, scenario_data, with_control=False):
        """
        求解振动响应(详细版本,使用真实场景数据)
        
        参数:
        t_span: 时间范围 (s)
        t_eval: 评估时间点 (s)
        scenario_data: 场景数据 DataFrame
        with_control: 是否启用主动控制
        
        返回:
        result: 求解结果字典
        """
        control_status = '有主动控制' if with_control else '无主动控制'
        print(f"\n开始求解振动响应 ({control_status})")
        
        # 提取数据并确保为数值类型
        t_data = pd.to_numeric(scenario_data.iloc[:, 0], errors='coerce').values  # 第一列是时间
        F_dist_data = pd.to_numeric(scenario_data.iloc[:, 1], errors='coerce').values  # 第二列是干扰力
        
        # 移除NaN值
        valid_mask = ~(np.isnan(t_data) | np.isnan(F_dist_data))
        t_data = t_data[valid_mask]
        F_dist_data = F_dist_data[valid_mask]
        
        # 干扰力插值函数
        F_dist_func = interp1d(t_data, F_dist_data, kind='linear', 
                              bounds_error=False, fill_value=0.0)
        
        # 初始条件
        initial_state = [self.y0, self.v0]
        
        if with_control and scenario_data.shape[1] >= 6:
            # 有控制情况,提取四组偏心块角度数据并确保为数值类型
            theta1_1_data = pd.to_numeric(scenario_data.iloc[:, 2], errors='coerce').values[valid_mask]
            theta1_2_data = pd.to_numeric(scenario_data.iloc[:, 3], errors='coerce').values[valid_mask]
            theta2_1_data = pd.to_numeric(scenario_data.iloc[:, 4], errors='coerce').values[valid_mask]
            theta2_2_data = pd.to_numeric(scenario_data.iloc[:, 5], errors='coerce').values[valid_mask]
            
            # 创建角位移插值函数
            theta_funcs = [
                interp1d(t_data, theta1_1_data, kind='linear', bounds_error=False, fill_value=0.0),
                interp1d(t_data, theta1_2_data, kind='linear', bounds_error=False, fill_value=0.0),
                interp1d(t_data, theta2_1_data, kind='linear', bounds_error=False, fill_value=0.0),
                interp1d(t_data, theta2_2_data, kind='linear', bounds_error=False, fill_value=0.0)
            ]
            
            # 计算角速度(数值微分)
            omega1_1_data = np.gradient(theta1_1_data, t_data)
            omega1_2_data = np.gradient(theta1_2_data, t_data)
            omega2_1_data = np.gradient(theta2_1_data, t_data)
            omega2_2_data = np.gradient(theta2_2_data, t_data)
            
            omega_funcs = [
                interp1d(t_data, omega1_1_data, kind='linear', bounds_error=False, fill_value=0.0),
                interp1d(t_data, omega1_2_data, kind='linear', bounds_error=False, fill_value=0.0),
                interp1d(t_data, omega2_1_data, kind='linear', bounds_error=False, fill_value=0.0),
                interp1d(t_data, omega2_2_data, kind='linear', bounds_error=False, fill_value=0.0)
            ]
            
            # 求解ODE
            sol = solve_ivp(
                lambda t, y: self.dynamics_with_control_detailed(t, y, F_dist_func, theta_funcs, omega_funcs),
                t_span, initial_state, t_eval=t_eval, 
                method='RK45', rtol=1e-8, atol=1e-10
            )
        else:
            # 无控制情况
            sol = solve_ivp(
                lambda t, y: self.dynamics_no_control(t, y, F_dist_func),
                t_span, initial_state, t_eval=t_eval, 
                method='RK45', rtol=1e-8, atol=1e-10
            )
        
        if not sol.success:
            raise RuntimeError(f"ODE求解失败: {sol.message}")
        
        # 提取结果
        t = sol.t
        y = sol.y[0]  # 位移
        v = sol.y[1]  # 速度
        
        # 计算加速度
        a = np.zeros_like(t)
        for i, ti in enumerate(t):
            if with_control and scenario_data.shape[1] >= 6:
                state = [y[i], v[i]]
                dydt = self.dynamics_with_control_detailed(ti, state, F_dist_func, theta_funcs, omega_funcs)
            else:
                state = [y[i], v[i]]
                dydt = self.dynamics_no_control(ti, state, F_dist_func)
            a[i] = dydt[1]
        
        # 计算横向振动指数
        T = t[-1] - t[0]
        vibration_index = np.trapz(a**2, t) / T
        
        print(f"求解完成,横向振动指数: {vibration_index:.6f} (m/s²)²")
        
        return {
            'time': t,
            'displacement': y,
            'velocity': v,
            'acceleration': a,
            'vibration_index': vibration_index,
            'with_control': with_control
        }
    
    def solve_vibration(self, t_span, t_eval, F_dist_data, theta_data=None, with_control=False):
        """
        求解振动响应
        
        参数:
        t_span: 时间范围 (s)
        t_eval: 评估时间点 (s)
        F_dist_data: 外部干扰力数据 [(t, F), ...]
        theta_data: 执行器角位移数据 [(t, theta), ...] (可选)
        with_control: 是否启用主动控制
        
        返回:
        result: 求解结果字典
        """
        control_status = '有主动控制' if with_control else '无主动控制'
        print(f"\n开始求解振动响应 ({control_status})")
        
        # 准备插值函数
        t_dist, F_dist = zip(*F_dist_data)
        F_dist_func = interp1d(t_dist, F_dist, kind='linear', 
                              bounds_error=False, fill_value=0.0)
        
        # 初始条件
        initial_state = [self.y0, self.v0]
        
        if with_control and theta_data is not None:
            # 有控制情况
            t_theta, theta = zip(*theta_data)
            theta_func = interp1d(t_theta, theta, kind='linear', 
                                 bounds_error=False, fill_value=0.0)
            
            # 计算角速度(数值微分)
            omega = np.gradient(theta, t_theta)
            omega_func = interp1d(t_theta, omega, kind='linear', 
                                 bounds_error=False, fill_value=0.0)
            
            # 求解ODE
            sol = solve_ivp(
                lambda t, y: self.dynamics_with_control(t, y, F_dist_func, theta_func, omega_func),
                t_span, initial_state, t_eval=t_eval, 
                method='RK45', rtol=1e-8, atol=1e-10
            )
        else:
            # 无控制情况
            sol = solve_ivp(
                lambda t, y: self.dynamics_no_control(t, y, F_dist_func),
                t_span, initial_state, t_eval=t_eval, 
                method='RK45', rtol=1e-8, atol=1e-10
            )
        
        if not sol.success:
            raise RuntimeError(f"ODE求解失败: {sol.message}")
        
        # 提取结果
        t = sol.t
        y = sol.y[0]  # 位移
        v = sol.y[1]  # 速度
        
        # 计算加速度
        a = np.zeros_like(t)
        for i, ti in enumerate(t):
            if with_control and theta_data is not None:
                state = [y[i], v[i]]
                dydt = self.dynamics_with_control(ti, state, F_dist_func, theta_func, omega_func)
            else:
                state = [y[i], v[i]]
                dydt = self.dynamics_no_control(ti, state, F_dist_func)
            a[i] = dydt[1]
        
        # 计算横向振动指数
        T = t[-1] - t[0]
        vibration_index = np.trapz(a**2, t) / T
        
        print(f"求解完成,横向振动指数: {vibration_index:.6f} (m/s²)²")
        
        return {
            'time': t,
            'displacement': y,
            'velocity': v,
            'acceleration': a,
            'vibration_index': vibration_index,
            'with_control': with_control
        }

def load_data_from_excel(file_path):
    """
    从Excel文件加载数据
    
    参数:
    file_path: Excel文件路径
    
    返回:
    data_dict: 包含所有数据的字典
    """
    print(f"正在加载数据文件: {file_path}")
    
    try:
        # 读取Excel文件的所有工作表
        excel_file = pd.ExcelFile(file_path)
        print(f"发现工作表: {excel_file.sheet_names}")
        
        data_dict = {}
        
        # 读取系统参数工作表
        if '系统参数' in excel_file.sheet_names:
            params_df = pd.read_excel(file_path, sheet_name='系统参数')
            print(f"系统参数工作表加载完成,形状: {params_df.shape}")
            
            # 提取参数值(参数在第三列,索引为2)
            params = {
                'mass': float(params_df.iloc[0, 2]),  # 车体质量
                'eccentric_mass': float(params_df.iloc[1, 2]),  # 偏心块质量
                'rotation_radius': float(params_df.iloc[2, 2]),  # 旋转半径
                'max_angular_velocity': float(params_df.iloc[3, 2]),  # 最大角速度
                'max_angular_acceleration': float(params_df.iloc[4, 2]),  # 最大角加速度
                'stiffness': float(params_df.iloc[5, 2]),  # 等效刚度
                'damping': float(params_df.iloc[6, 2])  # 等效阻尼
            }
            data_dict['parameters'] = params
            print(f"系统参数: {params}")
        
        # 读取场景1工作表
        if '场景1' in excel_file.sheet_names:
            scenario1_df = pd.read_excel(file_path, sheet_name='场景1')
            print(f"场景1工作表加载完成,形状: {scenario1_df.shape}")
            data_dict['scenario1'] = scenario1_df
        
        # 读取场景2工作表
        if '场景2' in excel_file.sheet_names:
            scenario2_df = pd.read_excel(file_path, sheet_name='场景2')
            print(f"场景2工作表加载完成,形状: {scenario2_df.shape}")
            data_dict['scenario2'] = scenario2_df
        
        return data_dict
    
    except Exception as e:
        print(f"数据加载失败: {e}")
        import traceback
        traceback.print_exc()
        # 如果无法读取Excel,返回模拟数据用于测试
        print("使用模拟数据进行测试...")
        return create_mock_data()

def create_mock_data():
    """
    创建模拟数据用于测试
    """
    print("创建模拟数据...")
    
    # 模拟系统参数
    params = {
        'mass': 1500.0,      # 车体质量 (kg)
        'stiffness': 50000.0, # 等效刚度 (N/m)
        'damping': 2000.0,    # 等效阻尼 (N·s/m)
        'eccentric_mass': 0.5, # 偏心质量 (kg)
        'rotation_radius': 0.1  # 旋转半径 (m)
    }
    
    # 时间序列
    t = np.linspace(0, 10, 1001)
    
    # 模拟场景1的干扰力(正弦波叠加)
    F_dist_scenario1 = 500 * np.sin(2*np.pi*1.5*t) + 200 * np.sin(2*np.pi*3.2*t) * np.exp(-0.1*t)
    
    # 模拟场景1的执行器角位移(控制策略)
    theta_scenario1 = 0.5 * np.sin(2*np.pi*1.5*t + np.pi) + 0.2 * np.sin(2*np.pi*3.2*t + np.pi/2)
    
    data_dict = {
        'parameters': pd.DataFrame([params]),
        'scenario1_disturbance': pd.DataFrame({'time': t, 'force': F_dist_scenario1}),
        'scenario1_control': pd.DataFrame({'time': t, 'theta': theta_scenario1})
    }
    
    return data_dict

def plot_results(result_no_control, result_with_control, save_path=None):
    """
    绘制对比结果图
    
    参数:
    result_no_control: 无控制时的结果
    result_with_control: 有控制时的结果
    save_path: 保存路径(可选)
    """
    fig, axes = plt.subplots(2, 2, figsize=(15, 10))
    fig.suptitle('车辆侧向振动控制效果对比分析', fontsize=16, fontweight='bold')
    
    # 位移对比
    axes[0, 0].plot(result_no_control['time'], result_no_control['displacement']*1000, 
                   'b-', linewidth=2, label='无主动控制')
    axes[0, 0].plot(result_with_control['time'], result_with_control['displacement']*1000, 
                   'r-', linewidth=2, label='有主动控制')
    axes[0, 0].set_xlabel('时间 (s)')
    axes[0, 0].set_ylabel('侧向位移 (mm)')
    axes[0, 0].set_title('侧向位移时程曲线')
    axes[0, 0].legend()
    axes[0, 0].grid(True, alpha=0.3)
    
    # 加速度对比
    axes[0, 1].plot(result_no_control['time'], result_no_control['acceleration'], 
                   'b-', linewidth=2, label='无主动控制')
    axes[0, 1].plot(result_with_control['time'], result_with_control['acceleration'], 
                   'r-', linewidth=2, label='有主动控制')
    axes[0, 1].set_xlabel('时间 (s)')
    axes[0, 1].set_ylabel('侧向加速度 (m/s²)')
    axes[0, 1].set_title('侧向加速度时程曲线')
    axes[0, 1].legend()
    axes[0, 1].grid(True, alpha=0.3)
    
    # 振动指数对比
    indices = [result_no_control['vibration_index'], result_with_control['vibration_index']]
    labels = ['无主动控制', '有主动控制']
    colors = ['blue', 'red']
    
    bars = axes[1, 0].bar(labels, indices, color=colors, alpha=0.7, edgecolor='black')
    axes[1, 0].set_ylabel('横向振动指数 (m/s²)²')
    axes[1, 0].set_title('横向振动指数对比')
    axes[1, 0].grid(True, alpha=0.3, axis='y')
    
    # 在柱状图上添加数值标签
    for bar, value in zip(bars, indices):
        height = bar.get_height()
        axes[1, 0].text(bar.get_x() + bar.get_width()/2., height + height*0.01,
                       f'{value:.4f}', ha='center', va='bottom', fontweight='bold')
    
    # 控制效果分析
    reduction_ratio = (result_no_control['vibration_index'] - result_with_control['vibration_index']) / result_no_control['vibration_index'] * 100
    
    axes[1, 1].text(0.1, 0.8, f"控制效果分析", fontsize=14, fontweight='bold', transform=axes[1, 1].transAxes)
    axes[1, 1].text(0.1, 0.6, f"无控制振动指数: {result_no_control['vibration_index']:.6f}", fontsize=12, transform=axes[1, 1].transAxes)
    axes[1, 1].text(0.1, 0.5, f"有控制振动指数: {result_with_control['vibration_index']:.6f}", fontsize=12, transform=axes[1, 1].transAxes)
    axes[1, 1].text(0.1, 0.4, f"振动抑制率: {reduction_ratio:.2f}%", fontsize=12, fontweight='bold', color='green', transform=axes[1, 1].transAxes)
    
    if reduction_ratio > 0:
        axes[1, 1].text(0.1, 0.2, "✓ 主动控制有效", fontsize=12, color='green', fontweight='bold', transform=axes[1, 1].transAxes)
    else:
        axes[1, 1].text(0.1, 0.2, "✗ 主动控制无效", fontsize=12, color='red', fontweight='bold', transform=axes[1, 1].transAxes)
    
    axes[1, 1].set_xlim(0, 1)
    axes[1, 1].set_ylim(0, 1)
    axes[1, 1].axis('off')
    
    plt.tight_layout()
    
    if save_path:
        plt.savefig(save_path, dpi=300, bbox_inches='tight')
        print(f"图表已保存至: {save_path}")
    
    plt.show()

def generate_report(result_no_control, result_with_control, save_path=None):
    """
    生成分析报告
    
    参数:
    result_no_control: 无控制时的结果
    result_with_control: 有控制时的结果
    save_path: 保存路径(可选)
    """
    reduction_ratio = (result_no_control['vibration_index'] - result_with_control['vibration_index']) / result_no_control['vibration_index'] * 100
    
    report = f"""
# 基于离心执行器的主动振动抑制分析报告

## 1. 问题概述
本报告针对大型车辆的侧向振动抑制问题,采用双电机离心执行器进行主动控制,
通过数值仿真分析了控制系统的振动抑制效果。

## 2. 仿真结果摘要

### 2.1 无主动控制工况
- 横向振动指数: {result_no_control['vibration_index']:.6f} (m/s²)²
- 最大侧向位移: {np.max(np.abs(result_no_control['displacement']))*1000:.3f} mm
- 最大侧向加速度: {np.max(np.abs(result_no_control['acceleration'])):.3f} m/s²

### 2.2 有主动控制工况  
- 横向振动指数: {result_with_control['vibration_index']:.6f} (m/s²)²
- 最大侧向位移: {np.max(np.abs(result_with_control['displacement']))*1000:.3f} mm
- 最大侧向加速度: {np.max(np.abs(result_with_control['acceleration'])):.3f} m/s²

### 2.3 控制效果评估
- 振动指数减少量: {result_no_control['vibration_index'] - result_with_control['vibration_index']:.6f} (m/s²)²
- 振动抑制率: {reduction_ratio:.2f}%
- 控制效果: {'有效' if reduction_ratio > 0 else '无效'}

## 3. 技术要点

### 3.1 数学模型
采用二阶线性常微分方程描述车体侧向动力学:
m·ÿ = F_dist + F_susp + F_ctrl

其中:
- F_susp = -k·y - c·ẏ (悬挂恢复力)
- F_ctrl = 8·m_e·r·ω²·sin(θ) (双执行器控制力)

### 3.2 数值方法
- 求解器: 四阶龙格-库塔法 (RK45)
- 相对误差容限: 1e-8
- 绝对误差容限: 1e-10

### 3.3 评价指标
横向振动指数定义为:
I_h = (1/T) ∫₀ᵀ a_y²(t) dt

## 4. 结论与建议

基于仿真分析结果,所设计的离心执行器主动控制系统在场景1的干扰条件下
{'能够有效抑制车辆的侧向振动' if reduction_ratio > 0 else '未能有效抑制车辆的侧向振动'}。
建议在实际应用中进一步优化控制策略,提高振动抑制效果。

---
报告生成时间: {pd.Timestamp.now().strftime('%Y-%m-%d %H:%M:%S')}
"""
    
    print(report)
    
    if save_path:
        with open(save_path, 'w', encoding='utf-8') as f:
            f.write(report)
        print(f"\n报告已保存至: {save_path}")
    
    return report

def main():
    """主函数"""
    print("="*60)
    print("基于离心执行器的主动振动抑制问题求解程序")
    print("问题A - 问题1:车体侧向运动建模与仿真分析")
    print("="*60)
    
    # 1. 加载数据
    data_file = "Problem A:Data.xlsx"
    data_dict = load_data_from_excel(data_file)
    
    # 2. 提取参数
    if 'parameters' in data_dict:
        params = data_dict['parameters']
    else:
        # 使用默认参数
        params = {
            'mass': 2000.0,
            'stiffness': 7225200.0,
            'damping': 3600.0,
            'eccentric_mass': 100.0,
            'rotation_radius': 0.2
        }
    
    # 3. 初始化模型
    model = VehicleVibrationModel(
        mass=params['mass'],
        stiffness=params['stiffness'],
        damping=params['damping'],
        eccentric_mass=params['eccentric_mass'],
        rotation_radius=params['rotation_radius']
    )
    
    # 4. 准备仿真参数
    t_span = (0, 10)
    t_eval = np.linspace(0, 10, 1001)
    
    # 5. 使用真实数据求解
    if 'scenario1' in data_dict:
        scenario1_data = data_dict['scenario1']
        
        print("\n" + "="*50)
        print("开始仿真分析...")
        
        # 6. 求解无控制情况
        result_no_control = model.solve_vibration_detailed(
            t_span, t_eval, scenario1_data, with_control=False
        )
        
        # 7. 求解有控制情况
        result_with_control = model.solve_vibration_detailed(
            t_span, t_eval, scenario1_data, with_control=True
        )
    else:
        print("未找到场景1数据,使用模拟数据...")
        # 使用模拟数据
        t = np.linspace(0, 10, 1001)
        F_dist = 500 * np.sin(2*np.pi*1.5*t) + 200 * np.sin(2*np.pi*3.2*t) * np.exp(-0.1*t)
        F_dist_data = list(zip(t, F_dist))
        theta = 0.5 * np.sin(2*np.pi*1.5*t + np.pi) + 0.2 * np.sin(2*np.pi*3.2*t + np.pi/2)
        theta_data = list(zip(t, theta))
        
        result_no_control = model.solve_vibration(
            t_span, t_eval, F_dist_data, with_control=False
        )
        result_with_control = model.solve_vibration(
            t_span, t_eval, F_dist_data, theta_data, with_control=True
        )
    
    # 8. 生成结果图表
    print("\n" + "="*50)
    print("生成结果图表...")
    plot_results(result_no_control, result_with_control, 'vibration_analysis_results.png')
    
    # 9. 生成分析报告
    print("\n" + "="*50)
    print("生成分析报告...")
    generate_report(result_no_control, result_with_control, 'vibration_analysis_report.md')
    
    # 10. 保存数值结果
    results_df = pd.DataFrame({
        'time': result_no_control['time'],
        'displacement_no_control': result_no_control['displacement'],
        'acceleration_no_control': result_no_control['acceleration'],
        'displacement_with_control': result_with_control['displacement'],
        'acceleration_with_control': result_with_control['acceleration']
    })
    results_df.to_csv('vibration_simulation_results.csv', index=False)
    print("数值结果已保存至: vibration_simulation_results.csv")
    
    print("\n" + "="*60)
    print("程序执行完成!")
    print("="*60)

if __name__ == "__main__":
    main()

\end{lstlisting}

\end{document}
